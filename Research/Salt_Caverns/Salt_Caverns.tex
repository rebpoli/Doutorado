\documentclass[a4paper]{article}
%\usepackage{simplemargins}

%\usepackage[square]{natbib}
\usepackage{amsmath}
\usepackage{amsfonts}
\usepackage{amssymb}
\usepackage{graphicx}
\usepackage{hyperref}
\hypersetup{
    colorlinks=true,
    linkcolor=blue,
    filecolor=magenta,      
    urlcolor=cyan,
    pdftitle={Overleaf Example},
    pdfpagemode=FullScreen,
    }
\begin{document}
\pagenumbering{gobble}

\normalsize
 \begin{center}
Literature assessment
 \end{center}
\Large
 \begin{center}
\vspace{30pt}
Salt caverns for gas storage / Numerical modelling

\hspace{10pt}

% Author names and affiliations
\large
Renato Poli (rep2656) \\

\hspace{10pt}

\end{center}

\hspace{10pt}

\normalsize

\section{Talk to Hassan Abadi (\href{https://apps.ualberta.ca/directory/person/dehghanp}{profile})}

He mentioned a sponsor: \href{https://www.respec.com/market/energy/caverns-hydrogen-underground-storage/}{RESPEC website}

\begin{itemize}
\item Impact of the cycles in stability
\item Low energy density of the caverns ($H_2$)
\item MIT - Mechanical Integrity Test (using $N_2$)
\item The brine in long contact with the cavern walls may impact stability?
\item Overview of projects worldwide? It seems that Canada has a well-set market on storage in caverns
\item Muri Dussault (???) - Kamy knows him
\item Adcore - utility company in Canada?
\item The key of the research is risk and regulation (form companies and governments)
\item Percolation of H2 in water saturated rock.
\item Flow in caverns is in open space and mainly gravitational
\item Well is a critical point! Cement-rock interface, how to ensure seal? Wellbore is a critical point of leakage
\item Tetsiana (Stanford) - talked about cement
\end{itemize}


\section{\cite{goulart20} Technology readiness assessment of ultra-deep salt caverns for carbon capture and storage in Brazil}

Paper funded by FAPESP RCGI (Research center for gas innovation - \href{https://sites.usp.br/rcgi/br/rgci_br/}{link}). Founder sponsors: Shell and Fapesp. One of the authors is Alvaro Maia, a ex-Senior Consultant in Petrobras (well drilling, geomechanics).

They study caverns of up to 150x450m, created by dissolution, with geomechanical simulations.

\begin{itemize}
\item Well is the critical element of an underground storage system, special attention to the cement.
\item May store 4 billion $Sm^3$
\item Used simulator COVES, developped by Alvaro Maia in the 1980s (See \cite{maia1984})
\end{itemize}

See also: \cite{abreu23}

\section{\cite{caglayan2020}Technical potential of salt caverns for hydrogen storage in Europe}


\section{\cite{li2021investigation} Investigation of thermal-mechanical effects on salt cavern during cycling loading}

Thermo-dependent salt creep. Thermo mechanical simulation. Pressure cycles, collapse and tensile fractures.

\section{\cite{coarita23} Hydromechanical modelling of salt caverns subjected to cyclic hydrogen injection and withdrawal}

2D models, to consder cyclic loading, with fundamental creep and plasticity. Storage depths varying from 350 to 1350m. They claim the deeper is more mechanically unstable. They find the caverns are feasible.

Regarding formation of hydrogen plume (I could not completely understand the physics yet):
\begin{quote}
In addition to the hydromechanical modelling, an analysis of the
hydrogen extent within the rock mass is carried out. Since water and
hydrogen are immiscible, we assume hereafter that the threshold
capillary pressure of rock salt is much higher than hydrogen pressure
within the cavern due to the nanometric size of pore throats and hence
no free gas phase will be flowing. Assuming hydrogen is a non-reactive
solute (no chemical reaction or sorption with salt), we use the following
mass transport equation to delineate the extent and mass of dissolved
hydrogen plume. Transport of dissolved H2 is only driven by advection
(Darcy’s law) and by diffusion, since mechanical dispersion is negligible
due to the low permeability of the porous medium.
\end{quote}

\section{\cite{zhao22} Feasibility analysis of salt cavern gas storage in extremely deep formation: A case study in China}



\section{Goals of the caverns}
\subsection{Energy storage}
\begin{itemize}
\item Regulate variations between renewable energy production and peak power demands (\cite{coarita23})
\end{itemize}

\subsection{Disposal}
%\begin{itemize}
%\end{itemize}

\section{The physics}

I see huge concern with geomechanical behavior, targeted to creep, collapse and tensile fracturing. The cyclic behavior of the pressure is a significant difference compared to oil drainage and waterflooding.

\begin{itemize}
\item Geomechanics: creep, thermal, tensile fracture, shear fractures
\item Chemical: dissolution
\item Cyclic operations
\item Temperature dependent creep
\item Plume extension ($H_2$ dissolved in water - percolation of $H_2$ in water saturated rock (?))
\item Heat transfer, especially considering cyclic operations
\item 
\end{itemize}

\newpage
\bibliographystyle{apalike}
\bibliography{Salt_Caverns}

\end{document}